% Options for packages loaded elsewhere
\PassOptionsToPackage{unicode}{hyperref}
\PassOptionsToPackage{hyphens}{url}
%
\documentclass[
]{article}
\usepackage{lmodern}
\usepackage{amssymb,amsmath}
\usepackage{ifxetex,ifluatex}
\ifnum 0\ifxetex 1\fi\ifluatex 1\fi=0 % if pdftex
  \usepackage[T1]{fontenc}
  \usepackage[utf8]{inputenc}
  \usepackage{textcomp} % provide euro and other symbols
\else % if luatex or xetex
  \usepackage{unicode-math}
  \defaultfontfeatures{Scale=MatchLowercase}
  \defaultfontfeatures[\rmfamily]{Ligatures=TeX,Scale=1}
\fi
% Use upquote if available, for straight quotes in verbatim environments
\IfFileExists{upquote.sty}{\usepackage{upquote}}{}
\IfFileExists{microtype.sty}{% use microtype if available
  \usepackage[]{microtype}
  \UseMicrotypeSet[protrusion]{basicmath} % disable protrusion for tt fonts
}{}
\makeatletter
\@ifundefined{KOMAClassName}{% if non-KOMA class
  \IfFileExists{parskip.sty}{%
    \usepackage{parskip}
  }{% else
    \setlength{\parindent}{0pt}
    \setlength{\parskip}{6pt plus 2pt minus 1pt}}
}{% if KOMA class
  \KOMAoptions{parskip=half}}
\makeatother
\usepackage{xcolor}
\IfFileExists{xurl.sty}{\usepackage{xurl}}{} % add URL line breaks if available
\IfFileExists{bookmark.sty}{\usepackage{bookmark}}{\usepackage{hyperref}}
\hypersetup{
  pdftitle={Seminários 1: Expectativas},
  hidelinks,
  pdfcreator={LaTeX via pandoc}}
\urlstyle{same} % disable monospaced font for URLs
\usepackage[margin=1in]{geometry}
\usepackage{color}
\usepackage{fancyvrb}
\newcommand{\VerbBar}{|}
\newcommand{\VERB}{\Verb[commandchars=\\\{\}]}
\DefineVerbatimEnvironment{Highlighting}{Verbatim}{commandchars=\\\{\}}
% Add ',fontsize=\small' for more characters per line
\usepackage{framed}
\definecolor{shadecolor}{RGB}{248,248,248}
\newenvironment{Shaded}{\begin{snugshade}}{\end{snugshade}}
\newcommand{\AlertTok}[1]{\textcolor[rgb]{0.94,0.16,0.16}{#1}}
\newcommand{\AnnotationTok}[1]{\textcolor[rgb]{0.56,0.35,0.01}{\textbf{\textit{#1}}}}
\newcommand{\AttributeTok}[1]{\textcolor[rgb]{0.77,0.63,0.00}{#1}}
\newcommand{\BaseNTok}[1]{\textcolor[rgb]{0.00,0.00,0.81}{#1}}
\newcommand{\BuiltInTok}[1]{#1}
\newcommand{\CharTok}[1]{\textcolor[rgb]{0.31,0.60,0.02}{#1}}
\newcommand{\CommentTok}[1]{\textcolor[rgb]{0.56,0.35,0.01}{\textit{#1}}}
\newcommand{\CommentVarTok}[1]{\textcolor[rgb]{0.56,0.35,0.01}{\textbf{\textit{#1}}}}
\newcommand{\ConstantTok}[1]{\textcolor[rgb]{0.00,0.00,0.00}{#1}}
\newcommand{\ControlFlowTok}[1]{\textcolor[rgb]{0.13,0.29,0.53}{\textbf{#1}}}
\newcommand{\DataTypeTok}[1]{\textcolor[rgb]{0.13,0.29,0.53}{#1}}
\newcommand{\DecValTok}[1]{\textcolor[rgb]{0.00,0.00,0.81}{#1}}
\newcommand{\DocumentationTok}[1]{\textcolor[rgb]{0.56,0.35,0.01}{\textbf{\textit{#1}}}}
\newcommand{\ErrorTok}[1]{\textcolor[rgb]{0.64,0.00,0.00}{\textbf{#1}}}
\newcommand{\ExtensionTok}[1]{#1}
\newcommand{\FloatTok}[1]{\textcolor[rgb]{0.00,0.00,0.81}{#1}}
\newcommand{\FunctionTok}[1]{\textcolor[rgb]{0.00,0.00,0.00}{#1}}
\newcommand{\ImportTok}[1]{#1}
\newcommand{\InformationTok}[1]{\textcolor[rgb]{0.56,0.35,0.01}{\textbf{\textit{#1}}}}
\newcommand{\KeywordTok}[1]{\textcolor[rgb]{0.13,0.29,0.53}{\textbf{#1}}}
\newcommand{\NormalTok}[1]{#1}
\newcommand{\OperatorTok}[1]{\textcolor[rgb]{0.81,0.36,0.00}{\textbf{#1}}}
\newcommand{\OtherTok}[1]{\textcolor[rgb]{0.56,0.35,0.01}{#1}}
\newcommand{\PreprocessorTok}[1]{\textcolor[rgb]{0.56,0.35,0.01}{\textit{#1}}}
\newcommand{\RegionMarkerTok}[1]{#1}
\newcommand{\SpecialCharTok}[1]{\textcolor[rgb]{0.00,0.00,0.00}{#1}}
\newcommand{\SpecialStringTok}[1]{\textcolor[rgb]{0.31,0.60,0.02}{#1}}
\newcommand{\StringTok}[1]{\textcolor[rgb]{0.31,0.60,0.02}{#1}}
\newcommand{\VariableTok}[1]{\textcolor[rgb]{0.00,0.00,0.00}{#1}}
\newcommand{\VerbatimStringTok}[1]{\textcolor[rgb]{0.31,0.60,0.02}{#1}}
\newcommand{\WarningTok}[1]{\textcolor[rgb]{0.56,0.35,0.01}{\textbf{\textit{#1}}}}
\usepackage{graphicx,grffile}
\makeatletter
\def\maxwidth{\ifdim\Gin@nat@width>\linewidth\linewidth\else\Gin@nat@width\fi}
\def\maxheight{\ifdim\Gin@nat@height>\textheight\textheight\else\Gin@nat@height\fi}
\makeatother
% Scale images if necessary, so that they will not overflow the page
% margins by default, and it is still possible to overwrite the defaults
% using explicit options in \includegraphics[width, height, ...]{}
\setkeys{Gin}{width=\maxwidth,height=\maxheight,keepaspectratio}
% Set default figure placement to htbp
\makeatletter
\def\fps@figure{htbp}
\makeatother
\setlength{\emergencystretch}{3em} % prevent overfull lines
\providecommand{\tightlist}{%
  \setlength{\itemsep}{0pt}\setlength{\parskip}{0pt}}
\setcounter{secnumdepth}{-\maxdimen} % remove section numbering

\title{Seminários 1: Expectativas}
\usepackage{etoolbox}
\makeatletter
\providecommand{\subtitle}[1]{% add subtitle to \maketitle
  \apptocmd{\@title}{\par {\large #1 \par}}{}{}
}
\makeatother
\subtitle{Darren Norris \& Fernanda Michalski\\
~\\
Resultados obtidos do questionário: ``Aulas remotas efetivas?''\\
Curso de Ciências Ambientais, Universidade Federal do Amapá (UNIFAP):
\url{https://www2.unifap.br/cambientais/}}
\author{}
\date{\vspace{-2.5em}16 dezembro 2020}

\begin{document}
\maketitle

\hypertarget{objetivo}{%
\subsection{Objetivo}\label{objetivo}}

Saber um pouco sobre as expectativas e recursos que os alunos
matriculados em Seminários 1 podem acessar atualmente. As respostas
servem para informar o desenvolvimento de aulas remotas efetivas.

Para isso fizemos sete perguntas aos alunos sobre suas expectativas e os
recursos disponíveis.

\hypertarget{achados-principais}{%
\subsection{Achados principais}\label{achados-principais}}

\begin{itemize}
\tightlist
\item
  A baixa taxa de respostas é preocupante. Mostrando que a maioria da
  turma matriculada em Seminários 1 não tem acesso e/ou interesse em
  participar ativamente com ensino remoto durante o calendário acadêmico
  suplementar.
\end{itemize}

Dos alunos que responderam, podemos afirmar que:-

\begin{itemize}
\item
  A maioria que respondeu tem acesso a internet e computador/laptop
  durante a maioria dos dias da semana. Mas, vamos desenvolver
  atividades em uma forma que todos podem acessar e participar. Por
  exemplo, considerando a instabilidade da internet nos não vamos
  apresentar aulas ao vivo prolongadas.
\item
  A maioria que responderam tem familiaridade com aplicativos de
  \href{https://www.microsoft.com/pt-br/microsoft-365}{Microsoft Office}
  e/ou \href{https://workspace.google.com/intl/pt-BR/}{Google workspace
  (antigo G-suite)} (mas não com aplicativos/ferramentas de
  \href{https://pt-br.libreoffice.org/}{LibreOffice} ).
\item
  A grande maioria que respondeu assistem videos online durante a
  semana.
\end{itemize}

\newpage

\hypertarget{resultados}{%
\subsection{Resultados}\label{resultados}}

Como primeira atividade do curso, fizemos sete perguntas aos alunos. De
um total de 22 alunos matriculados, 10 (45\%) responderam as perguntas.

Seguindo os princípios de transparência e
\href{https://book.fosteropenscience.eu/pt/02IntroducaoaCienciaAberta/04Investigacao_reprodutivel_e_analise_de_dados.html}{reprodutibilidade}
segue as analises e resultados obtidos do questionário: ``Aulas remotas
efetivas?''. O questionário foi disponibilizado na SIGAA entre 3 de
Novembro 2020 e 14 de Dezembro 2020.

Aqui é simplesmente para conhecimento. Apresentamos os resultados na
forma de gráficos, junto com o codigo, assim qualquer um que tem
interesse pode reproduzir. O texto é organizado em blocos de código
(caixas cinzas) e na sequencia com os resultados obtidos.

O objetivo não é de apresentar detalhes sobre os cálculos/métodos
estatísticas ou as funções no \href{https://cran.r-project.org/}{R}.

\hypertarget{pacotes}{%
\subsubsection{Pacotes}\label{pacotes}}

Deve Instalar os pacotes necessários antes de começar:

\begin{Shaded}
\begin{Highlighting}[]
\KeywordTok{install.packages}\NormalTok{(}\KeywordTok{c}\NormalTok{(}\StringTok{"plyr"}\NormalTok{,}\StringTok{"tidyverse"}\NormalTok{, }\StringTok{"readxl"}\NormalTok{, }\StringTok{"tidytext"}\NormalTok{, }\StringTok{"tm"}\NormalTok{,}\StringTok{"wordcloud2"}\NormalTok{))}
\end{Highlighting}
\end{Shaded}

Carregar pacotes:

\begin{Shaded}
\begin{Highlighting}[]
\KeywordTok{library}\NormalTok{(plyr)}
\KeywordTok{library}\NormalTok{(tidyverse)}
\KeywordTok{library}\NormalTok{(readxl)}
\KeywordTok{library}\NormalTok{(tidytext)}
\KeywordTok{library}\NormalTok{(tm)}
\KeywordTok{library}\NormalTok{(wordcloud2)}
\end{Highlighting}
\end{Shaded}

\hypertarget{carregar-arquivos-com-dados.}{%
\subsubsection{1) Carregar arquivos com
dados.}\label{carregar-arquivos-com-dados.}}

\hypertarget{carregar-dados-arquivo-do-excel}{%
\paragraph{1.1) Carregar dados (arquivo do
Excel)}\label{carregar-dados-arquivo-do-excel}}

Você deve obter o arquivo de Excel ``quest01.xlsx'' disponivel no SIGAA.
Dentro do arquivo as respostas obtidas estão apresentadas (em uma forma
anônima).

Agora avisar \href{https://cran.r-project.org/}{R} sobre onde ficar o
arquivo. O código abaixo vai abrir uma nova janela, e você deve buscar e
selecionar o arquivo ``quest01.xlsx'':

\begin{Shaded}
\begin{Highlighting}[]
\NormalTok{meuf <-}\StringTok{ }\KeywordTok{file.choose}\NormalTok{()}
\end{Highlighting}
\end{Shaded}

O código abaixo vai carregar os dados e criar um objeto ``df1''. Agora
temos dados com os resultados obtidos do primeiro questionário.

\begin{Shaded}
\begin{Highlighting}[]
\NormalTok{df1 <-}\StringTok{ }\NormalTok{readxl}\OperatorTok{::}\KeywordTok{read_excel}\NormalTok{(meuf)}
\end{Highlighting}
\end{Shaded}

\begin{Shaded}
\begin{Highlighting}[]
\CommentTok{# olhar conteudo do objecto "df1"}
\KeywordTok{str}\NormalTok{(df1)}
\end{Highlighting}
\end{Shaded}

\begin{verbatim}
## tibble [22 x 8] (S3: tbl_df/tbl/data.frame)
##  $ aid         : num [1:22] 1 2 3 4 5 6 7 8 9 10 ...
##  $ p01_expect  : chr [1:22] NA "Com esta matéria acredito que desenvolverei mais conhecimento sobre as temáticas que envolvem o curso, fazendo "| __truncated__ NA "A minha expectativa é que ao final da disciplina eu esteja mais apta a elaboração e realização da apresentação "| __truncated__ ...
##  $ p02_internet: chr [1:22] NA "raramente" NA "maioria dos dias" ...
##  $ p03_comp    : chr [1:22] NA "raramente" NA "sempre" ...
##  $ p04_gsuite  : chr [1:22] NA "zero" NA "menos de cinco vezes" ...
##  $ p05_ms      : chr [1:22] NA "zero" NA "menos de cinco vezes" ...
##  $ p06_vids    : chr [1:22] NA "zero" NA "mais de cinco vezes" ...
##  $ p07_lo      : chr [1:22] NA "zero" NA "menos de cinco vezes" ...
\end{verbatim}

\hypertarget{expectativas.}{%
\subsubsection{2) Expectativas.}\label{expectativas.}}

A primeira pergunta foi ``Considerando a situação atual (aulas remotas)
e os objetivos da disciplina, quais são as suas expectativas sobre
``Seminários 1''? Quais são os aprendizados que seria mais uteis para
você?".

As respostas variam nas informações apresentadas. Seguem três exemplos
ilustrativos:

\begin{itemize}
\item
  ``Com esta matéria acredito que desenvolverei mais conhecimento sobre
  as temáticas que envolvem o curso, fazendo com que assim eu tenha
  uma''luz" para a preparação de meu TCC e até mesmo na minha vida, já
  que com o período de crise que estamos vivendo, eu fiquei totalmente
  improdutiva em todos os sentidos acadêmicos e sociais."
\item
  ``A minha expectativa é que ao final da disciplina eu esteja mais apta
  a elaboração e realização da apresentação de trabalhos científicos,
  monografia, palestras, etc. Espero que as minhas dúvidas sejam sanadas
  e desde forma eu possa realizar o melhor de minha capacidade.''
\item
  ``Não sei o que esperar porém espero adquirir conhecimentos ao final
  da disciplina.''
\end{itemize}

\hypertarget{wordcloud}{%
\paragraph{2.1) Wordcloud}\label{wordcloud}}

Segue uma imagem composta por palavras usadas nas respostas, onde o
tamanho de cada palavra indica sua frequência (ou importância relativa).

\begin{Shaded}
\begin{Highlighting}[]
\CommentTok{#Codigo para obter um wordcloud}
\KeywordTok{set.seed}\NormalTok{(}\DecValTok{1234}\NormalTok{)}
\NormalTok{dfunt <-}\StringTok{ }\KeywordTok{unnest_tokens}\NormalTok{(df1, word, p01_expect)}
\CommentTok{# Excluindo palavras que podem ser consideradas irrelevantes para o conjunto de }
\CommentTok{# resultados a ser exibido}
\NormalTok{awords <-}\StringTok{ }\KeywordTok{c}\NormalTok{(}\StringTok{"agora"}\NormalTok{,}\StringTok{"so"}\NormalTok{, }\StringTok{"se"}\NormalTok{, }\StringTok{"sei"}\NormalTok{,}\StringTok{"nao"}\NormalTok{, }\StringTok{"ateh"}\NormalTok{, }\StringTok{"pre"}\NormalTok{, }\StringTok{"vez"}\NormalTok{, }\StringTok{"ateh"}\NormalTok{, }\StringTok{"soh"}\NormalTok{, }
            \StringTok{"ser"}\NormalTok{,}\StringTok{"creio"}\NormalTok{, }\StringTok{"sera"}\NormalTok{, }\StringTok{"serah"}\NormalTok{,}\StringTok{"cada"}\NormalTok{, }\StringTok{"eh"}\NormalTok{, }\StringTok{"real"}\NormalTok{, }\DecValTok{1}\OperatorTok{:}\DecValTok{20}\NormalTok{, }\StringTok{"forma"}\NormalTok{,}
            \StringTok{"posso"}\NormalTok{, }\StringTok{"possa"}\NormalTok{, }\StringTok{"é"}\NormalTok{, }\StringTok{"etc"}\NormalTok{, }\StringTok{"ja"}\NormalTok{, }\StringTok{"tal"}\NormalTok{, }\StringTok{"que"}\NormalTok{, }\StringTok{"desde"}\NormalTok{, }\StringTok{"acerca"}\NormalTok{, }
            \StringTok{"aulas"}\NormalTok{, }\StringTok{"espero"}\NormalTok{, }\StringTok{"trás"}\NormalTok{, }\StringTok{"quê", "}\NormalTok{ter}\StringTok{", "}\NormalTok{nessa}\StringTok{", "}\NormalTok{parece}\StringTok{", "}\NormalTok{assim}\StringTok{", }
\StringTok{            "}\NormalTok{expectativas}\StringTok{", "}\NormalTok{portanto}\StringTok{", "}\NormalTok{academicos}\StringTok{", "}\NormalTok{acadêmicos")}
\NormalTok{dwords <-}\StringTok{ }\NormalTok{tm}\OperatorTok{::}\KeywordTok{stopwords}\NormalTok{(}\StringTok{"portuguese"}\NormalTok{)}
\NormalTok{swords <-}\StringTok{ }\KeywordTok{unique}\NormalTok{(}\KeywordTok{c}\NormalTok{(awords, dwords))}
\NormalTok{selsw <-}\StringTok{ }\KeywordTok{which}\NormalTok{(dfunt}\OperatorTok{$}\NormalTok{word }\OperatorTok\StringTok{ }\NormalTok{swords)}
\NormalTok{dfunt <-}\StringTok{ }\NormalTok{dfunt[}\OperatorTok{-}\NormalTok{selsw, ]}
\NormalTok{dfunt <-}\StringTok{ }\NormalTok{dfunt }\OperatorTok
\StringTok{  }\KeywordTok{count}\NormalTok{(word, }\DataTypeTok{sort =} \OtherTok{TRUE}\NormalTok{, }\DataTypeTok{name=}\StringTok{"freq"}\NormalTok{)}
\end{Highlighting}
\end{Shaded}

Agora podemos visualizar as palavras mais comuns\ldots..

\begin{Shaded}
\begin{Highlighting}[]
\CommentTok{#Palavras escritas pelo menos duas vezes nas respostas}
\NormalTok{self <-}\StringTok{ }\KeywordTok{which}\NormalTok{(}\OperatorTok{!}\KeywordTok{is.na}\NormalTok{(dfunt}\OperatorTok{$}\NormalTok{word) }\OperatorTok{&}\StringTok{ }\NormalTok{dfunt}\OperatorTok{$}\NormalTok{freq}\OperatorTok{>}\DecValTok{1}\NormalTok{)}
\KeywordTok{wordcloud2}\NormalTok{(}\DataTypeTok{data =}\NormalTok{ dfunt[self, ])}
\end{Highlighting}
\end{Shaded}

\includegraphics{sem1_01exp_files/figure-latex/wordcloud-1.pdf}

\newpage

\hypertarget{recursos}{%
\subsubsection{3) Recursos}\label{recursos}}

Agora sobre disponibilidade e uso de recursos.

Olhamos três grupos distintos; todos importantes para aulas remotas.
Primeiro, sobre uso de internet e computadores, segundo sobre grupos de
aplicativos e por ultimo videos online.

\hypertarget{acesso-de-internet-e-computadorlaptop}{%
\paragraph{3.1) Acesso de internet e
computador/laptop}\label{acesso-de-internet-e-computadorlaptop}}

Segue as respostas para as perguntas ``Você tem acesso a internet
estável e segura?'' e ``Você tem acesso a um computador e/ou laptop?''.
As respostas incluíam uma das alternativas: não sei, nunca, raramente,
maioria dos dias, sempre.

\begin{Shaded}
\begin{Highlighting}[]
\CommentTok{# Codigo para Grupo 1: uso de internet e computador/laptop }
\CommentTok{# Organizar dados para fazer as figuras}
\NormalTok{nrep <-}\StringTok{ }\KeywordTok{length}\NormalTok{(}\KeywordTok{which}\NormalTok{(}\OperatorTok{!}\KeywordTok{is.na}\NormalTok{(df1}\OperatorTok{$}\StringTok{`}\DataTypeTok{p01_expect}\StringTok{`}\NormalTok{)))}

\NormalTok{dfeq<-}\StringTok{ }\KeywordTok{pivot_longer}\NormalTok{(df1[, }\KeywordTok{c}\NormalTok{(}\StringTok{'aid'}\NormalTok{,}\StringTok{'p02_internet'}\NormalTok{, }\StringTok{'p03_comp'}\NormalTok{)], }\OperatorTok{!}\NormalTok{aid, }\DataTypeTok{names_to =} \StringTok{"recursos"}\NormalTok{, }\DataTypeTok{values_to =} \StringTok{"resposta"}\NormalTok{)}
 
\NormalTok{dfeq}\OperatorTok{$}\NormalTok{recursos <-}\StringTok{ }\KeywordTok{factor}\NormalTok{(dfeq}\OperatorTok{$}\NormalTok{recursos)}
\KeywordTok{levels}\NormalTok{(dfeq}\OperatorTok{$}\NormalTok{recursos) <-}\StringTok{ }\KeywordTok{c}\NormalTok{(}\StringTok{"internet"}\NormalTok{, }\StringTok{"computador/laptop"}\NormalTok{)}
\CommentTok{#table(dfeq$resposta)}
\NormalTok{dfeq}\OperatorTok{$}\NormalTok{respostaf <-}\StringTok{ }\KeywordTok{factor}\NormalTok{(dfeq}\OperatorTok{$}\NormalTok{resposta, }
                            \DataTypeTok{levels =} \KeywordTok{c}\NormalTok{(}\StringTok{"raramente"}\NormalTok{, }
                                    \StringTok{"maioria dos dias"}\NormalTok{, }\StringTok{"sempre"}\NormalTok{))}
\CommentTok{#Obter resumo com proproçao e contagem}
\NormalTok{seleq <-}\StringTok{ }\KeywordTok{which}\NormalTok{(}\OperatorTok{!}\KeywordTok{is.na}\NormalTok{(dfeq}\OperatorTok{$}\NormalTok{resposta)) }
\NormalTok{dfeq_sum <-}\StringTok{ }\NormalTok{plyr}\OperatorTok{::}\KeywordTok{ddply}\NormalTok{(dfeq[seleq, ], .(recursos, respostaf), summarise, }
            \DataTypeTok{count_eq =} \KeywordTok{length}\NormalTok{(respostaf),}
            \DataTypeTok{prop_eq =} \KeywordTok{length}\NormalTok{(respostaf) }\OperatorTok{/}\StringTok{ }\NormalTok{nrep }
\NormalTok{            )}
\CommentTok{#Grafico}
\KeywordTok{ggplot}\NormalTok{(dfeq_sum, }\KeywordTok{aes}\NormalTok{(}\DataTypeTok{x =}\NormalTok{ respostaf, }\DataTypeTok{y =}\NormalTok{ prop_eq, }\DataTypeTok{fill=}\NormalTok{respostaf)) }\OperatorTok{+}
\StringTok{  }\KeywordTok{geom_col}\NormalTok{(}\DataTypeTok{colour=}\StringTok{"black"}\NormalTok{) }\OperatorTok{+}
\StringTok{  }\KeywordTok{scale_y_continuous}\NormalTok{(}\DataTypeTok{labels=}\NormalTok{scales}\OperatorTok{::}\NormalTok{percent) }\OperatorTok{+}
\StringTok{  }\KeywordTok{facet_wrap}\NormalTok{(}\OperatorTok{~}\NormalTok{recursos) }\OperatorTok{+}
\StringTok{  }\KeywordTok{labs}\NormalTok{(}\DataTypeTok{x =} \OtherTok{NULL}\NormalTok{, }\DataTypeTok{y =} \StringTok{"Respostas (%)"}\NormalTok{) }\OperatorTok{+}\StringTok{ }
\StringTok{  }\KeywordTok{scale_fill_brewer}\NormalTok{(}\DataTypeTok{palette =} \StringTok{"Greens"}\NormalTok{) }\OperatorTok{+}
\StringTok{  }\KeywordTok{theme}\NormalTok{(}\DataTypeTok{legend.position =} \StringTok{"none"}\NormalTok{) }\OperatorTok{+}
\StringTok{  }\KeywordTok{theme}\NormalTok{(}\DataTypeTok{axis.text.x =} \KeywordTok{element_text}\NormalTok{(}\DataTypeTok{angle =} \DecValTok{45}\NormalTok{, }\DataTypeTok{vjust =} \DecValTok{1}\NormalTok{, }\DataTypeTok{hjust=}\DecValTok{1}\NormalTok{))}
\end{Highlighting}
\end{Shaded}

\includegraphics{sem1_01exp_files/figure-latex/unnamed-chunk-8-1.pdf}

\hypertarget{uso-de-aplicativos-de-escrituxf3rio-produtividade}{%
\paragraph{3.2) Uso de aplicativos de ``escritório'' /
``produtividade''}\label{uso-de-aplicativos-de-escrituxf3rio-produtividade}}

Segue as respostas para as três perguntas ``Quantas vezes por semana
você usa aplicativos'' : G-suite, Microsoft Office e LibreOffice. As
respostas incluíam uma das alternativas: Nunca ouvir falar ; zero ;
menos de cinco vezes; mais de cinco vezes.

\begin{Shaded}
\begin{Highlighting}[]
\CommentTok{# Grupo 2: aplicativos}
\CommentTok{# Organizar dados para fazer as figuras}
\NormalTok{dfap <-}\StringTok{ }\KeywordTok{pivot_longer}\NormalTok{(df1[, }\KeywordTok{c}\NormalTok{(}\StringTok{'aid'}\NormalTok{,}\StringTok{'p04_gsuite'}\NormalTok{, }\StringTok{'p05_ms'}\NormalTok{,}\StringTok{'p07_lo'}\NormalTok{)], }\OperatorTok{!}\NormalTok{aid, }\DataTypeTok{names_to =} \StringTok{"recursos"}\NormalTok{, }\DataTypeTok{values_to =} \StringTok{"resposta"}\NormalTok{)}

\NormalTok{dfap}\OperatorTok{$}\NormalTok{recursos <-}\StringTok{ }\KeywordTok{factor}\NormalTok{(dfap}\OperatorTok{$}\NormalTok{recursos)}
\KeywordTok{levels}\NormalTok{(dfap}\OperatorTok{$}\NormalTok{recursos) <-}\StringTok{ }\KeywordTok{c}\NormalTok{(}\StringTok{"G-suite"}\NormalTok{, }\StringTok{"Microsoft Office"}\NormalTok{, }\StringTok{"LibreOffice"}\NormalTok{)}

\CommentTok{#table(dfap$resposta) # need to check how many of the possible levels}
\NormalTok{dfap}\OperatorTok{$}\NormalTok{respostaf <-}\StringTok{ }\KeywordTok{factor}\NormalTok{(dfap}\OperatorTok{$}\NormalTok{resposta, }
                            \DataTypeTok{levels =} \KeywordTok{c}\NormalTok{(}\StringTok{"nunca ouvir falar"}\NormalTok{, }\StringTok{"zero"}\NormalTok{, }
                                       \StringTok{"menos de cinco vezes"}\NormalTok{, }\StringTok{"mais de cinco vezes"}\NormalTok{) }
\NormalTok{)}
\CommentTok{#Resumo com contagem e proporçao}
\NormalTok{selap <-}\StringTok{ }\KeywordTok{which}\NormalTok{(}\OperatorTok{!}\KeywordTok{is.na}\NormalTok{(dfap}\OperatorTok{$}\NormalTok{resposta)) }
\NormalTok{dfap_sum <-}\StringTok{ }\NormalTok{plyr}\OperatorTok{::}\KeywordTok{ddply}\NormalTok{(dfap[selap, ], .(recursos, respostaf), summarise, }
            \DataTypeTok{count_ap =} \KeywordTok{length}\NormalTok{(respostaf),}
            \DataTypeTok{prop_ap =} \KeywordTok{length}\NormalTok{(respostaf) }\OperatorTok{/}\StringTok{ }\NormalTok{nrep }
\NormalTok{            )}

\CommentTok{#Grafico}
\KeywordTok{ggplot}\NormalTok{(dfap_sum, }\KeywordTok{aes}\NormalTok{(}\DataTypeTok{x =}\NormalTok{ respostaf, }\DataTypeTok{y =}\NormalTok{ prop_ap, }\DataTypeTok{fill=}\NormalTok{respostaf)) }\OperatorTok{+}
\StringTok{  }\KeywordTok{geom_col}\NormalTok{(}\DataTypeTok{color=}\StringTok{"black"}\NormalTok{) }\OperatorTok{+}
\StringTok{  }\KeywordTok{scale_y_continuous}\NormalTok{(}\DataTypeTok{labels=}\NormalTok{scales}\OperatorTok{::}\NormalTok{percent) }\OperatorTok{+}
\StringTok{  }\KeywordTok{facet_wrap}\NormalTok{(}\OperatorTok{~}\NormalTok{recursos) }\OperatorTok{+}
\StringTok{  }\KeywordTok{labs}\NormalTok{(}\DataTypeTok{x =} \OtherTok{NULL}\NormalTok{, }\DataTypeTok{y =} \StringTok{"Respostas (%)"}\NormalTok{) }\OperatorTok{+}\StringTok{ }
\StringTok{  }\KeywordTok{scale_fill_brewer}\NormalTok{(}\DataTypeTok{palette =} \StringTok{"Greens"}\NormalTok{) }\OperatorTok{+}
\StringTok{  }\KeywordTok{theme}\NormalTok{(}\DataTypeTok{legend.position =} \StringTok{"none"}\NormalTok{) }\OperatorTok{+}
\StringTok{  }\KeywordTok{theme}\NormalTok{(}\DataTypeTok{axis.text.x =} \KeywordTok{element_text}\NormalTok{(}\DataTypeTok{angle =} \DecValTok{45}\NormalTok{, }\DataTypeTok{vjust =} \DecValTok{1}\NormalTok{, }\DataTypeTok{hjust=}\DecValTok{1}\NormalTok{))}
\end{Highlighting}
\end{Shaded}

\includegraphics{sem1_01exp_files/figure-latex/unnamed-chunk-9-1.pdf}

\hypertarget{videos-online}{%
\paragraph{3.3) Videos online}\label{videos-online}}

Segue as respostas para a pergunta ``Quantas vezes por semana você
visualiza vídeos online?''.

\begin{Shaded}
\begin{Highlighting}[]
\CommentTok{# Grupo 3: videos}
\CommentTok{# Organizar dados para fazer as figuras}
\NormalTok{dfvid <-}\StringTok{ }\KeywordTok{pivot_longer}\NormalTok{(df1[, }\KeywordTok{c}\NormalTok{(}\StringTok{'aid'}\NormalTok{,}\StringTok{'p06_vids'}\NormalTok{)], }\OperatorTok{!}\NormalTok{aid, }\DataTypeTok{names_to =} \StringTok{"recursos"}\NormalTok{, }\DataTypeTok{values_to =} \StringTok{"resposta"}\NormalTok{)}

\NormalTok{dfvid}\OperatorTok{$}\NormalTok{recursos <-}\StringTok{ }\KeywordTok{factor}\NormalTok{(dfvid}\OperatorTok{$}\NormalTok{recursos)}
\KeywordTok{levels}\NormalTok{(dfvid}\OperatorTok{$}\NormalTok{recursos) <-}\StringTok{ }\KeywordTok{c}\NormalTok{(}\StringTok{"Videos online"}\NormalTok{)}

\CommentTok{#table(dfvid$resposta) # need to check how many of the possible levels}
\NormalTok{dfvid}\OperatorTok{$}\NormalTok{respostaf <-}\StringTok{ }\KeywordTok{factor}\NormalTok{(dfvid}\OperatorTok{$}\NormalTok{resposta, }
                            \DataTypeTok{levels =} \KeywordTok{c}\NormalTok{(}\StringTok{"zero"}\NormalTok{, }
                                       \StringTok{"menos de cinco vezes"}\NormalTok{, }\StringTok{"mais de cinco vezes"}\NormalTok{) }
\NormalTok{)}

\CommentTok{# resumo com contagem e proporçao}
\NormalTok{selvid <-}\StringTok{ }\KeywordTok{which}\NormalTok{(}\OperatorTok{!}\KeywordTok{is.na}\NormalTok{(dfvid}\OperatorTok{$}\NormalTok{resposta)) }
\NormalTok{dfvid_sum <-}\StringTok{ }\NormalTok{plyr}\OperatorTok{::}\KeywordTok{ddply}\NormalTok{(dfvid[selvid, ], .(recursos, respostaf), summarise, }
            \DataTypeTok{count_vid =} \KeywordTok{length}\NormalTok{(respostaf),}
            \DataTypeTok{prop_vid =} \KeywordTok{length}\NormalTok{(respostaf) }\OperatorTok{/}\StringTok{ }\NormalTok{nrep }
\NormalTok{            )}

\CommentTok{#Grafico}
\KeywordTok{ggplot}\NormalTok{(dfvid_sum, }\KeywordTok{aes}\NormalTok{(}\DataTypeTok{x =}\NormalTok{ respostaf, }\DataTypeTok{y =}\NormalTok{ prop_vid, }\DataTypeTok{fill=}\NormalTok{respostaf)) }\OperatorTok{+}
\StringTok{  }\KeywordTok{geom_col}\NormalTok{(}\DataTypeTok{colour=}\StringTok{"black"}\NormalTok{) }\OperatorTok{+}
\StringTok{  }\KeywordTok{scale_y_continuous}\NormalTok{(}\DataTypeTok{labels=}\NormalTok{scales}\OperatorTok{::}\NormalTok{percent) }\OperatorTok{+}
\StringTok{  }\KeywordTok{facet_wrap}\NormalTok{(}\OperatorTok{~}\NormalTok{recursos) }\OperatorTok{+}\StringTok{ }
\StringTok{  }\KeywordTok{scale_fill_brewer}\NormalTok{(}\DataTypeTok{palette =} \StringTok{"Greens"}\NormalTok{) }\OperatorTok{+}
\StringTok{  }\KeywordTok{theme}\NormalTok{(}\DataTypeTok{legend.position =} \StringTok{"none"}\NormalTok{) }\OperatorTok{+}
\StringTok{  }\KeywordTok{labs}\NormalTok{(}\DataTypeTok{x =} \OtherTok{NULL}\NormalTok{, }\DataTypeTok{y =} \StringTok{"Respostas (%)"}\NormalTok{)}
\end{Highlighting}
\end{Shaded}

\includegraphics{sem1_01exp_files/figure-latex/unnamed-chunk-10-1.pdf}

\end{document}
